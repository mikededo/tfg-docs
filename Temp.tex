% !TeX spellcheck = en_GB 

\documentclass[a4paper, 12pt, oneside]{book}

\usepackage[english]{babel}
\usepackage[utf8]{inputenc}
\usepackage[pagestyles]{titlesec}
\usepackage{titletoc}
\usepackage{enumitem}
\usepackage{fancyhdr}
\usepackage[a4paper, margin = 2.5cm, headheight=14pt]{geometry}
\usepackage[hidelinks, colorlinks = false]{hyperref}
\usepackage[bottom]{footmisc}
\usepackage{bookmark}
\usepackage{fontspec}
\usepackage{graphicx}
\usepackage[labelfont = bf]{caption}
\usepackage{etoolbox}
\usepackage{tabularx}
\usepackage{colortbl}
\usepackage{float}

\usepackage[table]{xcolor}

\title{
	\includegraphics[width=0.75\textwidth]{assets/logo.png}
	\\
	{\huge Hubbl, a gym bookings manager}
}
\author{Miquel de Domingo i Giralt}

\setmainfont{Inter}[BoldFont = Inter Bold]
\setmonofont[Scale=0.9]{JetBrains Mono}

\newfontfamily\semibf{Inter Semi Bold}

% Variable with the page that keeps the footer style
\def\footerstyle{- \fontsize{10pt}{10pt}\selectfont\thepage\ -}
\pagestyle{fancy}
% Remove the horizontal bar from the header
\renewcommand{\headrulewidth}{0pt}
% Clear everythihg
\fancyhf{}
% Set header style
\rhead{\small\textit{\nouppercase{\rightmark}}}
% Set the footer page number
\fancyfoot[C]{\footerstyle}
% Update the footer in chapter and other plain views
\fancypagestyle{plain}{%
    \renewcommand{\headrulewidth}{0pt}%
    \fancyhf{}%
    \fancyfoot[C]{\footerstyle}%
}

\setlength\parindent{0pt}
\setcounter{secnumdepth}{5}
\setcounter{tocdepth}{5}

% Tables format
\definecolor{rowColor}{RGB}{242, 242, 242}
\renewcommand{\arraystretch}{1.5}

\titleformat{\chapter}[hang]{\normalfont\huge\bfseries}{\thechapter. }{4pt}{\Huge}
\titlespacing{\chapter}{0pt}{-32pt}{12pt}

\begin{document}
\frontmatter
\maketitle
\mainmatter
\section{Deployment}
The deployment of the application would be the latest step once the 3 applications have been fineshed. Once again, due to the lack of time, I have not been able to finish all the features I had in mind before the application is deployed. Nevertheless, the current product is enough to be shown to potential clients, meaning that if I found any gym or company interested in such product, with what has been developed, I ensure it would be enough to attract them to the app.
\\[8pt]
Aside from that, a mind map of how the applications would work for the public, would be by accessing them from a public URL. The applications would be running in a cloud service, AWS to name one, and most likely deployed as docker images, which would fasten the deployment, and it would be easier to automate. However, due to lack of time and money, the application has not been deployed, meaning it can not be accessed from any device, other than the local machine. In order to run the application from the local machine, the installation manual can be found in the code repositiry, as well as in the documentation for each application, in the repository aswell.
\section{Results}
Since the initial design\footnote{The memory will include a copy of the Figma design, which will include all the assets that have been used in order to design the applicatio and to add images in the application.} had been really defined, the results of the application look nearly identic. However, when developing the application, due to the lack of time and other incovenients, some views may look a bit different.
\subsection{Core application results}
The following images are from the core application.
\subsection{Client application results}
The following images are from the client application.
\end{document}
% !TeX spellcheck = en_GB 

\documentclass[a4paper, 12pt, oneside]{book}

\usepackage[english]{babel}
\usepackage[utf8]{inputenc}
\usepackage[pagestyles]{titlesec}
\usepackage{titletoc}
\usepackage{enumitem}
\usepackage{fancyhdr}
\usepackage[a4paper, margin = 2.5cm, headheight=14pt]{geometry}
\usepackage[hidelinks, colorlinks = false]{hyperref}
\usepackage[bottom]{footmisc}
\usepackage{bookmark}
\usepackage{fontspec}
\usepackage{graphicx}
\usepackage[labelfont = bf]{caption}
\usepackage{etoolbox}
\usepackage{tabularx}
\usepackage{colortbl}
\usepackage{float}

\usepackage[table]{xcolor}

\title{
	\includegraphics[width=0.75\textwidth]{assets/logo.png}
	\\
	{\huge Hubbl, a gym bookings manager}
}
\author{Miquel de Domingo i Giralt}

\setmainfont{Inter}[BoldFont = Inter Bold]
\setmonofont[Scale=0.9]{JetBrains Mono}

\newfontfamily\semibf{Inter Semi Bold}

% Variable with the page that keeps the footer style
\def\footerstyle{- \fontsize{10pt}{10pt}\selectfont\thepage\ -}
\pagestyle{fancy}
% Remove the horizontal bar from the header
\renewcommand{\headrulewidth}{0pt}
% Clear everythihg
\fancyhf{}
% Set header style
\rhead{\small\textit{\nouppercase{\rightmark}}}
% Set the footer page number
\fancyfoot[C]{\footerstyle}
% Update the footer in chapter and other plain views
\fancypagestyle{plain}{%
    \renewcommand{\headrulewidth}{0pt}%
    \fancyhf{}%
    \fancyfoot[C]{\footerstyle}%
}

\setlength\parindent{0pt}
\setcounter{secnumdepth}{5}
\setcounter{tocdepth}{5}

% Tables format
\definecolor{rowColor}{RGB}{242, 242, 242}
\renewcommand{\arraystretch}{1.5}

\titleformat{\chapter}[hang]{\normalfont\huge\bfseries}{\thechapter. }{4pt}{\Huge}
\titlespacing{\chapter}{0pt}{-32pt}{12pt}

\begin{document}
\frontmatter
\maketitle
\mainmatter
\chapter{Future work}
It is barely ever true that a software application can be considered as \emph{finished}. Few are the cases in which the software does not need further development. The Hubbl application it will require of improvements and the addition of new features, so that the clients see more value in the organisation and application. Such thing is extremely common in product-bases software. Therefore, in general terms, the \emph{maintenance} and \emph{improvement} of the software will be the future work itself.
\\[8pt]
Aside from the general points, due to the lack of time and handwork (being me the only developer) it has not been possible to fulfill all the functional requirements. The following list contains what is left to be done:
\begin{enumerate}[label = -]
	\item \emph{Appointments}. Currently, it is not possible to view the list of appointments for an event or a gym zone. Therefore, the gym owner or any worker, are not able to see who is attending the gym zone\footnote{In the case of the events, it is possible to see how many clients have made an appoitment, yet it is not possible to know who.}. It is the same for the clients, they are not able to see their past and future appointments.
	      \newline
	      In the left navigation bar, a new tab will be added in both applications (core and client) which will allow the user to see the appointments made. In the case of the core application, it will display all the appointments made, while allowing the user to filter (by name, gym zone, event, event type and so on). Such view will require an enourmous amount of time for a singe developer to develop.
	\item \emph{Archival}. Archiving gym zones, events, clients and any item that can be deleted would allow the user to soft-delete items, meaning that such items could be recovered in the future. Currently, the server only supports hard-deletion, the opposite, as it requires less logic in the backed. However, the goal would be to provide the user with both utilities, being able to permanently remove or temporarily remove items.
	\item \emph{Update gym zones and virtual gyms}. It is currently not possible to edit the characteristics (name, description, open and close time, and so on) of a virtual gym nor a gym zone. I have not found the best place, in terms of user expercience, to locate such edition. Therefore, I have kept \emph{procastinating} the task, yet it would have a high priority in comparison to other tasks.
	\item \emph{Filtering}. The woker, trainer and client views will contain a lot of data which can be hard to visualise sometimes. That is the reason why providing filters for such views will add enourmous value to such features.
	\item \emph{Analytics}. Alongside the archiving feature, having analytics of the gym would provide tremendous value to the gym owner and their coworkers, but it requires a vast amount of time to be developed. Therefore, as I have stated in the FRE section, it was already something that I knew I wouldn't have time to develop, and it would be a \emph{nice-to-have} for the user. Such feature whould be added as another tab in the lateral menu, and it would only be accessible to the gym owner and to the workers with permissions to view them.
	\item \emph{Light/Dark theme}. It is currently possible in the server to change the theme from light to dark and vice-versa. However, since I did not have enought time to define a dark system design, I opted not to add it in the client. Nonetheless, it should be an easy to add feature, as the \emph{mui} library provides an easy tool for such feature.
	\item \emph{Client e2e}. Again, due to the lack of time and handwork, I have not been able to develop end-to-end tests for the client applications. I have little experience with developing such tests, and I would have required a vast amount of time to code consistent and valuable end-to-end tests.
	\item \emph{Deployment}. Finally, I would have wanted to deploy the applications to a web server so that they would be accessible to anyone. Such thing will not be possible until the application includes most of the above points, excluding analytics and archival.
\end{enumerate}
To sum up, even though the list may seem large, the application is fully functional and provides the basic utilities for the owners, workers and clients to properly organise themselves.
\end{document}